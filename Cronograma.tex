\documentclass[
	12pt,
	openright,
	twoside,
	a4paper,
	oneside,
	brazil,
	chapter=TITLE,
	%section=TITLE,
	%subsection=TITLE,
	sumario = tradicional
]{abntex2}

\usepackage[font=small,format=plain,labelfont=bf,up,margin=1cm]{caption}
\usepackage{amsmath}
\usepackage[utf8]{inputenc}
\usepackage[T1]{fontenc}
\usepackage[pdftex]{graphicx}
\usepackage{subfig}
\usepackage{float}
\usepackage[alf,abnt-etal-list=0, abnt-etal-cite=2, abnt-etal-text=it, abnt-repeated-author-omit=yes, abnt-emphasize=bf]{abntex2cite}
% alf - exibi as referencias em ordem alfabetica.
% abnt-full-initials=yes para mostrar nome completo nas referências
% abnt-etal-list=0 - foi utilizado para não permitir o uso de et al. nas referências.
% abnt-etal-cite=2 - mostra que pode ser feita a citação de até 2 autores no texto. Se numa referência existem 3 autores, usa-se evidentemente o et al., com a citação do primeiro autor.
% abnt-etal-text=it - exibi a escrita do et al nas referencias em italico.

\usepackage[table,xcdraw]{xcolor} % pacote para colorir células da tabela
\renewcommand{\arraystretch}{1.2} % espaçamento da linha na tabela

\usepackage{indentfirst} % justificar automaticamente os parágrafos
\usepackage{lscape}
\usepackage{import}
\usepackage{helvet}
\renewcommand{\familydefault}{\sfdefault}
\usepackage{lastpage}			% Usado pela Ficha catalográfica
\usepackage{color}				% Controle das cores
\usepackage[table]{xcolor}      % cores nas células das tabelas
\usepackage{microtype} 			% para melhorias de justificação

\usepackage{lipsum}             % gerador de lero lero

% Acrescentado para gerar as tabelas 22/07/2017
\usepackage[normalem]{ulem}
\useunder{\uline}{\ul}{}

%pacotes para desenhos e gráficos
\usepackage{tikz}
\usepackage{pgfplots}

%edição de escala em desenho

%%%%%%%%%%%%%%%%%%%%%%%%%%%%%%%%%%%%%%%

\usepackage{longtable}
\usepackage{empheq}
\usepackage{multirow}
\usepackage{booktabs}
\usepackage{pdfpages} %adicionar arquivos em pdf

\newcommand*\widefbox[1]{\fbox{\hspace{2em}#1\hspace{2em}}}
\newcommand{\head}[1]{\textnormal{\textbf{#1}}}
\newcolumntype{C}[2]{>{\centering\vspace{#2}\let\newline\\\arraybackslash\hspace{0pt}\vspace{#2}}m{#1}}

% Préambulo --------------------------------------------------------------------------------------------------------------------------

\newcommand{\universidade}{Universidade Federal do Vale do São Francisco }
\newcommand{\curso}{Engenharia Elétrica}
\newcommand{\graduacao}{Curso de Graduação em \curso}
\newcommand{\bancaum}{Professor Dr.,\\Faculdade}
\newcommand{\bancadois}{Professor Dr.,\\Faculdade}
\newcommand{\bibliotecario}{Inserir Bibliotecário}
\newcommand{\cpf}{000.000.000-00}
\newcommand{\siglaColegiado}{CENEL}
\newcommand{\siglaUniversidade}{UNIVASF}
\newcommand{\periodo}{2018.1}

% altera o tamanho da assinatura
\setlength{\ABNTEXsignwidth}{10cm}

\autor{Inserir seu nome}

\titulo{Inserir título}

\tipotrabalho{Trabalho de Conclusão de Curso}
\orientador[Orientador: ]{Dr. Professor Fulano}
%\coorientador[Coorientador: Dr.]{Siclano}
\instituicao{\universidade \par \curso}
\local{Juazeiro - BA}
\data{2018}

\preambulo{
	Cronograma e Plano de Trabalho para tese de Conclusão de Curso apresentado como requisito parcial para obtenção do título de Bacharel em \curso, pela \universidade (UNIVASF).
}

% alterando o aspecto da cor azul.
\definecolor{blue}{RGB}{41,5,195}

% Configurações de aparência do PDF final.
\makeatletter
\hypersetup {
	pdftitle		= {\@title},
	pdfauthor		= {\@author},
	pdfsubject      = {\imprimirpreambulo},
	pdfcreator		= {LaTeX with abnTeX2},
	pdfkeywords 	= {Palavra chave 1}{Palavra chave 2}{Palavra chave 3},
	colorlinks		= true,			% false: boxed links; true: colored links
	linkcolor		= blue,         % cor links internos.
	citecolor		= blue,       	% cor links referencias bibliograficas.
	filecolor		= magenta,		% cor links de arquivos.
	urlcolor		= blue,			% Cor url.
	bookmarksdepth	= 4
}
\usepackage{smartdiagram}
\makeatother

% compila o sumário
\makeindex

% Altera as margens padrões
\pagenumbering{arabic}

\setlrmarginsandblock{3cm}{2cm}{*}
\setulmarginsandblock{3cm}{2cm}{*}
\checkandfixthelayout

% Espaçamentos  entre linhas e parágrafos 
% O tamanho do parágrafo é dado por:
\setlength{\parindent}{1.5cm}
% Controle do espaçamento entre um parágrafo e outro:
\setlength{\parskip}{0.2cm}  % tente também \onelineskip

% Retira espaço extra obsoleto entre as frases.
\frenchspacing

\begin{document}

	\selectlanguage{brazil}

	% ELEMENTOS PRE-TEXTUAIS

	\pretextual

	% Capa
	
	\begin{capa}
		\center

		\vspace*{-2.5cm}

		\begin{figure}[!htbp]
		 	\centering
		 	\includegraphics[scale=0.08]{./figuras/univasf.pdf}
		\end{figure}

		\ABNTEXchapterfont {\large \textbf {\MakeUppercase{\universidade}\\ \MakeUppercase{\graduacao} }}

		\vspace*{2cm}

		\ABNTEXchapterfont {\Large \imprimirautor}\\
		\vspace*{3cm}
		\ABNTEXchapterfont {\Large \textbf{CRONOGRAMA E PLANO DE TRABALHO}}
    
        \vspace*{3cm}
		%\vfill

		\begin{center}
			\ABNTEXchapterfont \bfseries {\Large \imprimirtitulo}
		\end{center}

		\vfill

		{\large \imprimirlocal} \\
		{\large \imprimirdata}

		\vspace*{1cm}
	\end{capa}

	
	% Folha de Rosto
	\include{elementos-pretextuais/folhaderosto}
	\include{elementos-pretextuais/Termo}
	

	% Índice (Sumário)
	\pdfbookmark[0]{\contentsname}{toc}
    \tableofcontents*
    \cleardoublepage
	
	%%%%%%%%%%%%%%%%%%% ELEMENTOS TEXTUAIS %%%%%%%%%%%%%%%%
	\chapterstyle{BlackBox}
	\textual
    
    \chapter{Cronograma}

O cronograma é a representação gráfica do tempo que será utilizado para a confecção de um trabalho ou projeto. Sendo assim, para o presente trabalho serão executadas atividades preferencialmente conforme a sequência: pesquisa do tema, pesquisa bibliográfica, coleta de informações, apresentação dos dados, discussão dos dados, elaboração do trabalho, entrega do trabalho e defesa de tese.

\begin{table}[H]
\centering
\caption{Detalhamento do Cronograma}
\label{my-label}
\resizebox{\textwidth}{!}{%
\begin{tabular}{|l|c|c|c|c|c|c|c|}
\hline
\multicolumn{1}{|c|}{\textbf{ATIVIDADES}} & \multicolumn{1}{l|}{\textbf{ABRIL}} & \textbf{MAIO} & \textbf{JUNHO} & \textbf{JULHO} & \textbf{AGOSTO} & \multicolumn{1}{l|}{\textbf{SETEMBRO}} & \multicolumn{1}{l|}{\textbf{OUTUBRO}} \\ \hline
Pesquisa do Tema & X & X &  &  &  &  &  \\ \hline
Pesquisa Bibliográfica &  & X & X &  &  &  &  \\ \hline
\begin{tabular}[c]{@{}l@{}}Coleta de Informações\\ (se for o caso)\end{tabular} &  &  &  & X &  &  &  \\ \hline
\begin{tabular}[c]{@{}l@{}}Apresentação e Discussão\\ do Dados\end{tabular} &  &  &  & X & X &  &  \\ \hline
Elaboração do Trabalho &  &  & X & X & X &  &  \\ \hline
Entrega do Trabalho &  &  &  &  &  & X &  \\ \hline
Defesa da Tese &  &  &  &  &  &  & X \\ \hline
\end{tabular}%
}
\end{table}

É válido salientar que a imposição de ordem sequencial não impede o desenvolvimento de atividades paralelas, tal como pesquisa bibliográfica e coleta de informações, como pode ser visto no fluxograma da Figura \ref{fluxograma}.

\begin{figure}[H]
    \centering
    \resizebox{7cm}{!}{
    \smartdiagram[priority descriptive diagram]{
    Pesquisa do Tema,
    Pesquisa Bibliográfica e Coleta de Informações,
    Resultados,
    Elaboração do Trabalho,
    Defesa da Tese}
    }
    \caption{Ordem do cronograma}
    \label{fluxograma}
\end{figure}
    \chapter{Plano de Trabalho}

\section{Local do Estudo}
O estudo será realizado no Laboratório de Engenharia Elétrica da Universidade Federal do Vale do São Francisco.

\section{Período do Estudo}
O estudo será realizado no período de aproximadamente cinco meses, correspondente ao intervalo entre Abril de 2018 e Setembro de 2018.

\section{Levantamento Bibliográfico}
Etapa destinada à pesquisa acerca dos tópicos relacionados ao desenvolvimento do trabalho.

\section{Análise de Retorno Financeiro}
Etapa em que será realizado simulação financeira para cálculo de \textit{payback} do investimento necessário para produção de energia elétrica.

\section{Análise dos Resultados}
Etapa destinada a Análise dos Resultados e possíveis ajustes das etapas anteriores.

\section{Elaboração do Documento}
Esta última etapa será constituída da elaboração do TCC.

\section{Defesa Pública}
Ocorrerá a defesa do TCC.


    %%%%%%%%%%%%%%%%%%% FIM %%%%%%%%%%%%%%%%

\end{document}
